%%%%%%%%%%%%%%%%%%%%%%%%%%%%%%%%%%%%%%%%%%%%%%%%%%%%%%%%%%%%%%%
%
% Welcome to Overleaf --- just edit your LaTeX on the left,
% and we'll compile it for you on the right. If you give
% someone the link to this page, they can edit at the same
% time. See the help menu above for more info. Enjoy!
%
% Note: you can export the pdf to see the result at full
% resolution.
%
%%%%%%%%%%%%%%%%%%%%%%%%%%%%%%%%%%%%%%%%%%%%%%%%%%%%%%%%%%%%%%%
% Title: Block diagram of Third order noise shaper in Compact Disc Players
% Author: Ramón Jaramillo
\documentclass[tikz,14pt,border=10pt]{standalone}
%%%<
\usepackage{verbatim}
%%%>
\begin{comment}
:Title: Block diagram of Third order noise shaper in Compact Disc Players
:Tags: Diagrams;Block diagrams;Electrical engineering
:Author: Ramón Jaramillo
:Slug: noise-shaper

This is a block diagram of a third-order noise shaper, circuit located inside
of a compact disc player.

Source: A fundamental introduction to the Compact Disc Player (1994),
located in http://www.tc.umn.edu/~erick205/Papers/3011Paper.pdf, page 17
by Grant M. Erickson
\end{comment}
\usepackage{textcomp}
\usetikzlibrary{shapes,arrows}
\begin{document}
% Definition of blocks:
\tikzset{%
  block/.style    = {draw, thick, rectangle, minimum height = 3em,
    minimum width = 3em},
  sum/.style      = {draw, circle, node distance = 2cm}, % Adder
  input/.style    = {coordinate}, % Input
  output/.style   = {coordinate} % Output
}
% Defining string as labels of certain blocks.
\newcommand{\suma}{\Large$+$}
\newcommand{\inte}{$\displaystyle \int$}
\newcommand{\derv}{\huge$\frac{d}{dt}$}

\begin{tikzpicture}[auto, thick, node distance=2cm, >=triangle 45]
% Drawing Encoder
% Step 1
\draw
	node at (1,0)[right=-3.2mm]{\Large \textopenbullet}
	node at (1,0)[input, name=hx0] {}
	node [block, right of=hx0] (rnnx1) {
      \begin{tabular}{cc}
      RNN \\
      Encoder
      \end{tabular}
    }
    node [block, below of=rnnx1] (embedx1) {Embed}
    node at (3,-4)[above=-3.2mm]{\Large \textopenbullet}
    node at (3,-4)[input, name=x1] {};

	\draw[->](hx0) -- node {$h_{x_0}$}(rnnx1);
    \draw[->](embedx1) -- node {$x_1$} (rnnx1);
    \draw[->](x1) -- node {It}(embedx1);

% Step 2
\draw
	node at (6,0)[block] (rnnx2) {
      \begin{tabular}{cc}
      RNN \\
      Encoder
      \end{tabular}
    }
    node [block, below of=rnnx2] (embedx2) {Embed}
    node at (6,-4)[above=-3.2mm]{\Large \textopenbullet}
    node at (6,-4)[input, name=x2] {};

    \draw[->](rnnx1) -- node {$h_{x_1}$}(rnnx2);
    \draw[->](embedx2) -- node {$x_2$} (rnnx2);
    \draw[->](x2) -- node {is}(embedx2);

% Step 3
\draw
	node at (9,0)[block] (rnnx3) {
      \begin{tabular}{cc}
      RNN \\
      Encoder
      \end{tabular}
    }
    node [block, below of=rnnx3] (embedx3) {Embed}
    node at (9,-4)[above=-3.2mm]{\Large \textopenbullet}
    node at (9,-4)[input, name=x3] {};

    \draw[->](rnnx2) -- node {$h_{x_2}$}(rnnx3);
    \draw[->](embedx3) -- node {$x_3$} (rnnx3);
    \draw[->](x3) -- node {an}(embedx3);

% Step 4
\draw
	node at (12,0)[block] (rnnx4) {
      \begin{tabular}{cc}
      RNN \\
      Encoder
      \end{tabular}
    }
    node [block, below of=rnnx4] (embedx4) {Embed}
    node at (12,-4)[above=-3.2mm]{\Large \textopenbullet}
    node at (12,-4)[input, name=x4] {};

    \draw[->](rnnx3) -- node {$h_{x_3}$}(rnnx4);
    \draw[->](embedx4) -- node {$x_4$} (rnnx4);
    \draw[->](x4) -- node {Apple}(embedx4);

\draw [color=gray,thick](1.75,-3) rectangle (13.25,1);
\node at (1.75,0.8) [above=5mm, right=0mm] {\textsc{Encoder}};

% Attention
\draw
	node at (7.5, 2.5)[block, text width=10cm,align=center](attention) {\Large{Attention over $\textbf{H}_x$}}
	node at (4.1,0) {\textbullet}
    node at (7.1,0) {\textbullet}
    node at (10.1,0) {\textbullet};
\draw[->] (rnnx1) -- (4.1,0) -- (4.1,1.5) -- (attention);
\draw[->] (rnnx2) -- (7.1,0) -- (7.1,1.5) -- (attention);
\draw[->] (rnnx3) -- (10.1,0) -- (10.1,1.5) -- (attention);
\draw[->] (rnnx4) -- (13.1,0) -- (13.1,1.5) node [right] {$h_{x_4}$} -- (attention);

% Drawing Decoder
% Step 1
\draw
	node at (0,8)[right=-3.2mm]{\Large \textopenbullet}
	node at (0,8)[input, name=hy0] {}
	node [block, right of=hy0] (rnny1) {
      \begin{tabular}{cc}
      RNN \\
      Decoder
      \end{tabular}
    }
    node at (1.5,4)[above=-3.2mm]{\Large \textopenbullet}
    node at (1.5,4)[input, name=y1] {}
    node [block, above of=y1] (embedy1) {Embed}
    node [above of=rnny1] (word1) {Es};

	\draw[->](hy0) -- node {$h_{y_0}$}(rnny1);
    \draw[->](embedy1) -- node [right] {$y_1$} (1.5,7.45);
    \draw[->](y1) -- node {start\_token}(embedy1);
    \draw[->](attention) -- (2.5,4) -- node [right]{$\phi_1$}(2.5,7.45);
    \draw[->](rnny1) -- (word1);
% Step 2
\draw
	node at (5,8)[block] (rnny2) {
      \begin{tabular}{cc}
      RNN \\
      Decoder
      \end{tabular}
    }
    node at (4.5,6) [block] (embedy2) {Embed}
    node [above of=rnny2] (word2) {ist};

    \draw[->](rnny1) -- node {$h_{y_1}$}(rnny2);
    \draw[->](embedy2) -- node {$y_2$} (4.5,7.45);
    \draw[->](word1) -- (3.1,10) -- (3.1,6) -- (embedy2);
    \draw[->](attention) -- (5.5,4) -- node [right]{$\phi_2$}(5.5,7.45);
    \draw[->](rnny2) -- (word2);
% Step 3
\draw
	node at (8,8)[block] (rnny3) {
      \begin{tabular}{cc}
      RNN \\
      Decoder
      \end{tabular}
    }
    node at (7.5,6) [block] (embedy3) {Embed}
    node [above of=rnny3] (word3) {eine};

    \draw[->](rnny2) -- node {$h_{y_2}$}(rnny3);
    \draw[->](embedy3) -- node {$y_3$} (7.5,7.45);
    \draw[->](word2) -- (6.1,10) -- (6.1,6) -- (embedy3);
    \draw[->](attention) -- (8.5,4) -- node [right]{$\phi_3$}(8.5,7.45);
    \draw[->](rnny3) -- (word3);

% Step 4
\draw
	node at (11,8)[block] (rnny4) {
      \begin{tabular}{cc}
      RNN \\
      Decoder
      \end{tabular}
    }
    node at (10.5,6) [block] (embedy4) {Embed}
    node [above of=rnny4] (word4) {Apfel};

    \draw[->](rnny3) -- node {$h_{y_3}$}(rnny4);
    \draw[->](embedy4) -- node {$y_4$} (10.5,7.45);
    \draw[->](word3) -- (9.1,10) -- (9.1,6) -- (embedy4);
    \draw[->](attention) -- (11.5,4) -- node [right]{$\phi_4$}(11.5,7.45);
    \draw[->](rnny4) -- (word4);

% Step 5
\draw
	node at (14,8)[block] (rnny5) {
      \begin{tabular}{cc}
      RNN \\
      Decoder
      \end{tabular}
    }
    node at (13.5,6) [block] (embedy5) {Embed}
    node [above of=rnny5] (word5) {end\_token};

    \draw[->](rnny4) -- node {$h_{y_4}$}(rnny5);
    \draw[->](embedy5) -- node {$y_5$} (13.5,7.45);
    \draw[->](word4) -- (12.1,10) -- (12.1,6) -- (embedy5);
    \draw[->](attention) -- (14.5,4) -- node [right]{$\phi_5$}(14.5,7.45);
    \draw[->](rnny5) -- (word5);

\draw [color=gray,thick](0.75,5) rectangle (15.2,10.5);
\node at (0.75,10.3) [above=5mm, right=0mm] {\textsc{Decoder}};
\end{tikzpicture}
\end{document}
